\documentclass[UTF8]{ctexart}
\usepackage{graphicx}
\usepackage{geometry}
\usepackage{latexsym}
\title{计算机体系结构 实验1}
\date{\today}
\author{PB18000203汪洪韬}
\begin{document}
	\maketitle
	\section{实验说明}
	本次实验是 RV32I Core 设计的铺垫。我们提供了一个样例 RV32I Core 的设计图,参考设计图,理解每条指令需要的数据通路,以及相应的控制信号,并回答相应问题。
	
	\section{待实现指令}
	RISC-V 32bit 整型指令集(除去FENCE,FENCE.I,CSR,ECALL 和 EBREAK 指令)
	
	\section{实验报告}
	\subsection{描述执行一条 XOR 指令的过程(数据通路、控制信号等)。}
	数据通路:IM取指令译码得到rs1、rs2和rd的地址,在Reg中得到rs1和rs2的值,在EX段进行计算,最后写回rd。
	
	控制信号:Alusrc选择两个寄存器的值,AluControl选择XOR,RegWrite选择有效,LoadNpc选择ALU输出,MemtoReg选择Alu输出,其余控制信号无效。
	
	\subsection{描述执行一条 BEQ 指令的过程(数据通路、控制信号等)。}
	数据通路:IM取指令译码得到rs1、rs2的地址和SB类imm的值,在Reg中得到rs1和rs2的值,在EX段计算,若BrE有效则PC\_IN=PCF+BrT,跳转到目标地址;若BrE无效则PC\_IN=PCF+4。
	
	控制信号:ImmType选择SB,Alusrc选择两个寄存器的值,AluControl选择相减,BranchType选择BEQ,若相等则BrE为有效,LoadNpc为ALU输出,其余信号无效。
	
	\subsection{描述执行一条 LHU 指令的过程(数据通路、控制信号等)。}
	数据通路:IM取指令译码得到,rs地址和I类imm的值,在Reg中得到rs的值,在EX段计算,然后将结果作为地址在MEM段取值,最后在WB段将16位数符号扩展为32位写回寄存器。
	
	控制信号:RegWrite有效,MemToReg选择ImmType选择mem数据,Alusrc1选择rs的值,AluSrc2选择imm的值,AluControl选择LH,LoadNpc为ALU输出,LoadByteSelect为LH,其余信号无效。
	
	\subsection{如果要实现 CSR 指令(csrrw,csrrs,csrrc,csrrwi,csrrsi,csrrci),设计图中还需要增加什么部件和数据通路?给出详细说明。}
	\begin{itemize}
		\item 控制信号:增加CSR读写使能信号;增加AluSrc2中CSR数据选择信号;增加立即数扩展中的CSR扩展信号;
		\item ID段:添加CSR寄存器;立即数扩展模块支持CSR扩展;
		\item EX段:Alu操作数2的选择中增加CSR寄存器的值;
		\item WB段:写回CSR寄存器。
	\end{itemize}

	\subsection{Verilog 如何实现立即数的扩展?}
	\begin{center}
		\includegraphics[width = .5\textwidth]{inst.JPG}\\
	\end{center}

	\subsection{如何实现 Data Cache 的非字对齐的 Load 和 Store?}
	对于LOAD指令,将由ALU计算的结果的低两位清零,将结果作为对齐后的地址进行LOAD;对于STORE,根据WE的值来确定地址的非对齐的位数,然后进行STORE操作。
	
	\subsection{ALU 模块中,默认 wire 变量是有符号数还是无符号数?}
	无符号数
	
	\subsection{简述BranchE信号的作用。}
	条件分支中,若EX段中两个寄存器的值相同则BranchE信号有效,控制NPC Generator使得$PC\_IN=PCF+BrT$,跳转到目标地址;若BrE无效则$PC\_IN=PCF+4$。
	
	\subsection{NPC Generator 中对于不同跳转 target 的选择有没有优先级?}
	有,BrE、JalrE大于JalD,因为前两者更先执行(EX段指令)。
	
	\subsection{Harzard 模块中,有哪几类冲突需要插入气泡,分别使流水线停顿几个周期?}
	\begin{itemize}
		\item 跳转分支指令,停顿2个周期
		\item 装载-使用型,停顿1个周期
	\end{itemize}

	\subsection{Harzard 模块中采用静态分支预测器,即默认不跳转,遇到 branch 指令时,如何控制 flush 和 stall 信号?}
	若不跳转则无需flush和stall;若跳转则需要flush IF/ID和ID/EX,并将stall置0,停止下面两条语句的执行。
	\subsection{0 号寄存器值始终为 0,是否会对 forward 的处理产生影响?}
	写入0号寄存器总是被丢弃,提供了常量0和写入丢弃的场所;利用 addi x0, x0, 0来实现nop; 当某条运算指令写x0时,不对后续指令转发运算结果,而是转发0。
	
	\section{实验收获}
	回顾了组成原理实验中的流水线设计的大部分内容,同时对RISC-V流水线设计的基本数据通路和控制信号有了一定的认识,对一些新的指令和原先指令的数据通路的变化有了了解,对新的控制信号的作用和产生也有了了解,收获很大。
\end{document}